:insert
%
% File Name     : folien.tex
% Purpose       :
% Creation Date : 28-03-2014
% Last Modified : Tue 01 Nov 2016 05:30:32 PM CET
% Created By    :
%


\\documentclass[xcolor=table]{beamer}

\\usepackage[utf8]{inputenc}

\\usepackage[ngerman]{babel}

\\usepackage{graphicx}

\\usepackage{biblatex}

\\usepackage{pgfplots}

\\usepackage{ragged2e}

\\usepackage{tikz}

\\usetikzlibrary{mindmap,trees}

\\usepackage{verbatim}


\\usetheme{Darmstadt}


\\definecolor{lavander}{cmyk}{0,0.48,0,0}

\\definecolor{violet}{cmyk}{0.79,0.88,0,0}

\\definecolor{burntorange}{cmyk}{0,0.52,1,0}

\\def\lav{lavander!90}

\\def\oran{orange!30}

\\tikzstyle{peers}=[draw,circle,violet,bottom color=\lav, top color= white, text=violet,minimum width=10pt]

\\tikzstyle{superpeers}=[draw,circle,burntorange, left color=\oran, text=violet,minimum width=20pt]

\\tikzstyle{legendsp}=[rectangle, draw, burntorange, rounded corners, thin,bottom color=\oran, top color=white, text=burntorange, minimum width=2.5cm]

\\tikzstyle{legendp}=[rectangle, draw, violet, rounded corners, thin, bottom color=\lav, top color= white, text= violet, minimum width= 2.5cm] 

\\tikzstyle{legend_general}=[rectangle, rounded corners, thin, burntorange, fill= white, draw, text=violet, minimum width=2.5cm, minimum height=0.8cm]

%\\logo{\includegraphics[scale=1]{logo.png}}

\\title{Title}

\\author{Author}

\\date{\today}


\\begin{document}

\\maketitle

\\setcounter{tocdepth}{1}


\\frame{\frametitle{Agenda}\tableofcontents}


\\section{Einleitung}

\\begin{frame}{\secname}

\\begin{tikzpicture}

\\path[mindmap,concept color=black,text=white]
node[concept] {Computer Science} [clockwise from=0]
child[concept color=green!50!black] {
node[concept] {practical} [clockwise from=90]
child { node[concept] {algorithms} }
child { node[concept] {data structures} }
child { node[concept] {pro\-gramming languages} }
child { node[concept] {software engineer\-ing} }
}  

child[concept color=blue] {
node[concept] {applied} [clockwise from=-30]
child { node[concept] {databases} }
child { node[concept] {WWW} }
}

child[concept color=red] { node[concept] {technical} }
child[concept color=orange] { node[concept] {theoretical} };

\\end{tikzpicture}

\\end{frame}


\\subsection{Funktionsweise}

\\begin{frame}{\secname : \subsecname}

    \\begin{itemize}

        \\item[] 

    \\end{itemize}

\\end{frame}  


\\begin{frame}[fragile]

\\frametitle{Bar Graphs}

\\begin{tikzpicture}[scale=0.85]

\\begin{axis}[xbar,tick align=outside,
    width=11cm,
    height=8cm,
    bar width={10pt},
    enlargelimits=0.13,
    nodes near coords,
    nodes near coords align=horizontal,
    point meta=x * 1, % The displayed number.
    xlabel=\textbf{Number of Finals Won},
    xtick={0,5,...,35},
    ytick={1,...,13},
    yticklabels={Kerry,Laois,London,Waterford,Clare,Offaly,
    Galway,Wexford,Dublin,Limerick,Tipperary,Cork,Kilkenny}
]

\\addplot[draw=blue,fill=blue!15]coordinates{(1,1) (1,2) (1,3) (2,4) (3,5) (4,6) (4,7) (6,8) (6,9) (7,10) (26,11) (30,12) (33,13)};

\\end{axis}

\\end{tikzpicture}

\\end{frame}

\\begin{frame}

\\begin{tikzpicture}[auto, thick]

% Place super peers and connect them

\\foreach \place/\name in {{(0,-1)/a}, {(2,0)/b}, {(2,2)/c}, {(0,2)/d}, {(-2,0)/e}}

\\node[superpeers] (\name) at \place {};

\\foreach \source/\dest in {a/b, a/c, a/d, b/c, c/d,a/e,d/e}

\\path (\source) edge (\dest);

%

% Place normal peers

\\foreach \pos/\i in {above left of/1, left of/2, below left of/3}

\\node[peers, \pos = e] (e\i) {};

\\foreach \speer/\peer in {e/e1,e/e2,e/e3}

\\path (\speer) edge (\peer);

%

\\foreach \pos/\i in {above right of/1, right of/2, below right of/3}

\\node[peers, \pos =b ] (b\i) {};

\\foreach \speer/\peer in {b/b1,b/b2,b/b3}

\\path (\speer) edge (\peer);

%

\\node[peers, above of=d] (d1){};

\\path (d) edge (d1);

%

\\foreach \pos/\i in {below left of/1, below of/2}

\\node[peers, \pos =a ] (a\i) {};

\\foreach \speer/\peer in {a/a1,a/a2}

\\path (\speer) edge (\peer);

%%%%%%%%

% Legends

\\node[legendsp] at (5,0) {\small{Super-peers}};

\\node[legendp] at (5,2) {\small{Normal peers}};

\\node[legend_general] at (0,4) {\small{\textsc{Skype-topology}}};

\\end{tikzpicture}

\\end{frame}


\\subsection{Bewertung}

\\begin{frame}{\secname : \subsecname}

	\\begin{columns}

        \\column{.5\textwidth}

{
        \\setbeamercolor{block title}{bg=green, fg=white}

        \\begin{block}{Vorteile}

            \\begin{itemize}

                \\item 

                \\item

                \\item

            \\end{itemize}

        \\end{block}	
}

        \\column{.5\textwidth}

{
        \\setbeamercolor{block title}{bg=red, fg=white}

        \\begin{block}{Nachteile}

            \\begin{itemize}

                \\item

                \\item

                \\item 

            \\end{itemize}

        \\end{block}	
}

    \\end{columns}		

\\end{frame}  


\\section{Zusammenfassung}

\\begin{frame}{\secname}

    Tabelle Gegenueberstellung...

\\end{frame}


\\begin{frame}[plain]

    \\begin{center}
    
    \\huge Vielen Dank.

    \\end{center}

\\end{frame}


\\appendix

\\begin{frame}[allowframebreaks]{Quellen}

    \\begin{thebibliography}{AUTOR, ERSCHEINUNGSJAHR}

    \\tiny

        \\bibitem[AUTOR, ERSCHEINUNGSJAHR]{AUTORJAHR}
            AUTOR.

            \\newblock TITEL.

            \\newblock {\em VERLAG, ERSCHEINUNGSJAHR, ISBN}.


        \\bibitem[AUTOR]{2014}
            AUTOR,

            \\TITEL

            \\newblock \scalebox{.9}{\url{https://www.seite.com/de/datei.pdf}}

            \\newblock {\em besucht am 00.00.0000}.


    \\end{thebibliography}

\\end{frame}

\\end{document}
